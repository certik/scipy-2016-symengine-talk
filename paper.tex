%% Submissions for peer-review must enable line-numbering
%% using the lineno option in the \documentclass command.
%%
%% Preprints and camera-ready submissions do not need
%% line numbers, and should have this option removed.
%%
%% Please note that the line numbering option requires
%% version 1.1 or newer of the wlpeerj.cls file.

\documentclass[fleqn,10pt,lineno,numbers]{wlpeerj} % for journal submissions
% \documentclass[fleqn,10pt]{wlpeerj} % for preprint submissions

\usepackage{lmodern}
\usepackage[T1]{fontenc}
\usepackage[utf8]{inputenc}
\usepackage[scaled=0.8]{DejaVuSansMono}

\usepackage{hyperref}
\usepackage{graphicx}
\usepackage[all]{xy}
\usepackage{amsmath}
\usepackage{caption}
\graphicspath{ {images/} }

% Makes quote characters in monospace font not be curly
\usepackage{upquote}

\usepackage{amsmath}
\usepackage{url}
\usepackage{hyperref}

% this is required for all the \url{} commands in the bib file
%\usepackage{hyperref}

% for nice units
\usepackage{siunitx}

% for images: png, pdf, etc
\usepackage{graphicx}

% for nice table formatting, i.e., /toprule, /midrule, etc
\usepackage{booktabs}

% to allow for \verb++ declarations in captions.
\usepackage{cprotect}

% to allow usage of \mathbb symbols
\usepackage{amssymb}

\usepackage{longtable}

\usepackage{listings}

\newcommand\email[1]{\href{mailto:#1}{#1}}
\title{SymEngine}

\author[1]{Ondřej Čertík}%
\author[2]{Isuru Fernando}%
\affil[1]{Los Alamos National Laboratory, Los Alamos, NM 87545 (\email{certik@lanl.gov}).}%
\affil[2]{University of Moratuwa, Bandaranayake Mawatha, Katubedda, Moratuwa 10400, Sri Lanka (\email{isuru.11@cse.mrt.ac.lk}).}%


\keywords{symbolic, C++11, computer algebra system}

\begin{abstract}
SymEngine is a symbolic manipulation library written in C++ that aims to be the fastest
in symbolic manipulation (opensource or commercial). SymEngine is compatible with SymPy,
and can be used from many languages (Python, Ruby, Julia, ...). We will present the
current status of development, how things are implemented internally, why we chose C++,
benchmarks, and examples of usage from Python (SymPy and Sage), Ruby and Julia.
\end{abstract}

\begin{document}

\flushbottom
\maketitle
\thispagestyle{empty}

\section{Introduction}

SymPy is a widely used symbolic manipulation library in Python. While SymPy turns out to be
very useful for many applications, one of the long term problems with SymPy is that the
speed might be insufficient when handling of very large expressions is required. Another
problem with SymPy is that due to being written in Python, it can be cumbersome to use
from other languages like Julia, Ruby, JavaScript or C++, because it requires, say, a Python
to C++ bridge, which might not always be robust and which inflicts additional overhead.

We will show benchmarks against other computer algebra systems (opensource and commercial)
as well as examples of usage. We will present a roadmap on porting SymPy on top of SymEngine.
We will talk about why we chose C++ and what rules to follow so that the code cannot have an
undefined behavior in Debug mode (thus providing similar ease of development as one is used
to from Python).



\section{Architecture}
\label{sec:architecture}

\section{Features}
\label{sec:features}

\section{Numerics}
\label{sec:numerics}

\section{Conclusion and future work}
\label{sec:conclusion}


\section{Acknowledgements}
\label{sec:acknowledgements}


\bibliography{paper}

\end{document}
